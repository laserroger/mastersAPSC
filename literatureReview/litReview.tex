\documentclass[12pt]{article}

\usepackage{hyperref}
\usepackage{setspace}
\usepackage[round]{natbib}
\bibliographystyle{plainnat}

\newcommand{\superscript}[1]{\ensuremath{^{\textrm{#1}}}}
\newcommand{\subscript}[1]{\ensuremath{_{\textrm{#1}}}}

\begin{document}

\title{Literature Review}

\author{Andrew Valencik}

\date{\today}

\maketitle

\tableofcontents

%%%

\begin{doublespacing}

%%% Begin Introduction
\section{Introduction}

%%% General intro
% - Nuclear Science
% - Database
% - Datamining

%%% Nuclear Science
% - Radiation
% - Selected technique: Semiconductor detectors
% - Selected technique: FPGAs

%%% Knowledge Discovery
% - Intro to KD and DM
% - Selected technique: Clustering and Classification
% - Selected technique: Sentiment Analysis

\section{Nuclear Science}
The study of radiation and the structure of the atom is more thann 100 years old.

\subsection{Radiation}
Radiation can be emitted through multiple processes involving many interactions. The radiation carries energy away from the parent nuclei or atom. Thus systems in an excited state may emit radiation to decay to their ground state or another unstable state. 

\subsubsection{Alpha Decay}
Alpha, α, decay is a type of nuclear decay prevalent in heavy nuclei, in which an alpha particle (helium nucleus), is ejected from the parent nuclide. Alpha emitters are useful in \href{http://cs.smu.ca/~andrew/files/ugthesis/SMU-Thesis-v.1.0.1.pdf}{calibrating} nuclear detectors as they are effectively monoenergetic in terms of their energy resolution. There is a species change associated with alpha decay depicted in the following equation.
The mechanism for alpha decay is Coulomb barrier penetration via \href{https://en.wikipedia.org/wiki/Quantum_tunneling}{quantum tunnelling}. The barrier penetration probability is needed to describe the relationship between energy and halflife. Alpha decay can happen alongside larger \href{https://en.wikipedia.org/wiki/Nuclear_fission}{fission breakup} reactions.
% img: [](/img/alphaDecayEqn.gif)
% img: [Alpha Decay^[[Wiki Commons: Alpha Decay Image](http://commons.wikimedia.org/wiki/File:Alpha_Decay.svg)]](http://upload.wikimedia.org/wikipedia/commons/thumb/7/79/Alpha_Decay.svg/200px-Alpha_Decay.svg.png)

\subsubsection{Beta Decay}
There are three types of beta, β, decay, beta-minus decay, beta-plus decay, and electron capture. Both beta-minus decay, and beta-plus decay involve the ejection of a beta particle (an \href{http://en.wikipedia.org/wiki/Electron}{electron} or \href{http://en.wikipedia.org/wiki/Positron}{positron}) and a \href{http://en.wikipedia.org/wiki/Neutrino}{neutrino} (or \href{http://en.wikipedia.org/wiki/Antineutrino#Antineutrinos}{antineutrino}). Thus beta decay also produces a species change as described below. \href{https://en.wikipedia.org/wiki/Electron_capture}{Electron capture}, converts a nuclear proton to a neutron by absorping an orbiting electron.
The neutrinos have a very low interaction probability with matter and are thus not detected directly. The beta particle and the neutrino take away a small amount of momentum from the system, resulting in a minute recoil energy on the daughter nucleus. This recoil energy is generally well below the ionization threshold of the detecting volume and therefore goes unmeasured.
Beta emitters are good sources of fast electrons. A pure beta emitter is one that decays directly to a ground state. The energy spectrum of beta decay is continuous. If monoenergetic electrons are desired, a system that undergoes \href{https://en.wikipedia.org/wiki/Internal_conversion}{internal conversion} is desirable. The two processes are worth mentioning together as some beta emitters will undergo internal conversion after a beta decay event.
% img: [](/img/betaDecayEqn.gif)
% img: \href{http://commons.wikimedia.org/wiki/File:Beta-minus_Decay.svg}{Beta Decay^[[Wiki Commons: Beta Decay Image}]](http://upload.wikimedia.org/wikipedia/commons/thumb/a/aa/Beta-minus_Decay.svg/200px-Beta-minus_Decay.svg.png)

\subsubsection{Nuclear Fission}
Nuclear fission is the break up of an unstable nucleus into smaller pieces. These pieces can be quite large (by definition larger than an alpha particle) and can be unstable themselves. Nuclei that undergo fission have a distribution of emitted masses, not a single finite mass as in alpha decay. Nuclear fission is usually accompanied by alpha decay and neutron emission.

\subsubsection{Gamma Radiation}
An unstable nuclear state can decay by emitting a high energy photon called a gamma, γ, ray. Because the gamma ray is uncharged it is usually detected by observing the effects of its interaction with a detecting volume. In this form of radiation there is no change of species. 
Gamma rays have enough energy to ionize some materials. It is through this process, and the production of \href{https://en.wikipedia.org/wiki/Secondary_emmission}{secondary emission} that gamma rays are detected. Gamma rays can follow beta decay, in which case they have halflife properties from the parent, but reflect energy levels of the daughter nucleus.
Gamma rays can also be produced by an annihilation event where an electron and positron annihilate and produce two gamma rays of 0.511MeV in opposite directions. Because of this, beta-plus decay, which releases positrons, is often accompanied by 0.511MeV gamma rays.
% img: [](/img/gammaDecayEqn.gif)
% img: \href{http://commons.wikimedia.org/wiki/File:Gamma_Decay.svg}{Gamma Radiation^[[Wiki Commons: Gamma Radiation Image}]](http://upload.wikimedia.org/wikipedia/commons/thumb/c/c2/Gamma_Decay.svg/200px-Gamma_Decay.svg.png)

\subsubsection{Neutron Radiation}
Another form of uncharged radiation, neutrons can be detected by the use of an absorber material which emits characteristic energy or ions that are then detected separately.
Absorption of thermal neutrons can prompt the emission of gamma rays from the nucleus.
%TODO ref to knoll
A good discussion on neutron radiation applications and concepts can be found in Knoll's Radiation Detection and Measurement.

\subsubsection{Propagation}
Over time the energy release from a radioactive source is usually isotropic. In an experimental setup, a solid angle portion may be incident on a [semiconductor detector](semiconductorDetectors). To avoid energy loss through interactions occurring between the source and the detector, the experimental setup is typically under vacuum. 
The penetration depth varies among radiation types. Photons propagate through the \href{https://en.wikipedia.org/wiki/Electromagnetic_field}{electromagnetic field} at the \href{https://en.wikipedia.org/wiki/Speed_of_light}{speed of light}. Other particulate radiation has a velocity dependent on its energy. Typical energies of alpha decay can be stopped by a piece of paper, beta decay by thin films of aluminum, and gamma rays by lead shielding.


\pagebreak
\subsection{Semiconductor Detectors}   %%%%%%%%%%%%%%%%%%%%%%%%%%%%%%%%%%
A wide range of detectors are available for various forms of radiation.
Papers in this section focus on a specific implementation of a detector and usually describe the architectural setup at a high level.

\subsubsection{Immediate Sources of Energy Loss}

%% Entrance Window
In the case of a HPGe, High Purity Germanium, detector the radiation must pass through a protecting window. 
Materials such as Beryllium or Aluminium may be used in ~1mm thicknesses.\footnote{\url{http://www.physics.fsu.edu/courses/Spring12/PHY3802L/exp4822/aptec_EGPC-.pdf}} 
Their low densities and atomic masses allow some radiation to pass through with little energy loss.
\\

%% Dead Layer
The dead layer is a section of the detecting volume that does not contribute to detecting. 
Incident radiation may lose energy here, and no detection event is registered. 
This effect may be significant for heavy charged particles or weakly penetrating radiation.
The thickness of this layer can be reduced with an applied voltage called a bias voltage.
\\


\subsubsection{Structure of Detector Volume}

%% Crystal Lattice
The radiation begins interacting with the crystal lattice. 
High energy photons impact electrons in the lattice and depart some energy via ionization, atomic excitation, and bremsstrahlung emission. 
\\

%% p-n Junctions
A p-n junction is boundary between a p-type semiconductor and a n-type semiconductor material. 
A n-type semiconductor has been doped to have an excess of electrons. 
Similarly, a p-type semiconductor is implanted with a dopant such that it has an abundance of holes. 
A displacement of charge carries occurs at a p-n junction. 
This is the result of a diffusion force felt by charge carriers near the junction. 
The electrons are attracted from the n-type material towards the p-type material through the Coulomb interaction with the excess hole in the p-type semiconductor. 
Eventually a equilibrium charge distribution is reached, at which there is a potential difference across the junction opposing further diffusion.
\\


\subsubsection{Interactions}
Photoelectric absorption: The photon transfers its energy to an atom, knocking out an electron, typically from the K shell. 
The electron, a photoelectron, has kinetic energy equal to the the photon minus the original binding energy of the electron, and a very small amount of recoil energy.

%% Secondary Emissions
The output signal of a detector is reliant on collecting a charge associated with the radiation's interaction with the detecting volume. 
These interactions can excite other particles in the volume that contribute to the overall charge.
Detector materials are designed to favourably allow these interactions and produce a large amount of secondary emissions. 
\\


\subsubsection{Detector Operation}
We make use of a simplified detector model to describe some of the processes involved in detector operation.
Consider some incident radiation on our detecting volume. 
By some process the radiation interacts with the detecting medium and deposits energy which produces a net charge. 
A voltage is applied across the detecting volume, this biases the volume creating an electric field. 
When charge carriers are produced by an interaction, the electron-hole pairs migrate apart, in opposite directions according to the field. 
Without a bias the charge carriers are immobile and will not produce a current at the collection sites.

%% Pulse Mode
A detector can be run in current mode, mean square voltage mode, or pulse mode. 
The pulse mode operation of a detector will be the focus of this study as it best enables the detection of individual radiation quanta by providing information on amplitude and timing.
\\

%% Charge Collection
The bias applied to the detecting volume promotes charge carrying electrons into the conduction band and draws them to the collection sites. 
Holes are also collected, but an opposite collection site.
\\

%% Dead Time
A detector's dead time is the duration after a radiation event before the detector can correctly respond to another event.
There are two models for dead time, paralyzable and nonparalyzable. 
If consecutive events in the detector extend the dead time, it is said to be paralyzable. 
In this behaviour a count rate higher than the inverse of the dead time will completely stop the detection of new events. 
Each radiation interact will restart the dead time of the system and thus no events will be recorded. 
In the nonparalyzable model the dead time is not restarted by an event occurring during the dead time. 
Thus in the limiting case the count rate will be no higher than the inverse of the dead time.
So it is always possible that an 'event' contains multiple incidences of radiation.
\\



{\large\textbf{\cite{Leviner201466}}}
The first N-type segmented Germanium detector with 85\% $^{76}\mbox{Ge}$ enrichment is characterized and evaluated for use in the Majorana collaboration.
The segmented enriched germanium array (SEGA) is applicable to neutrinoless double beta-decay because of the strong energy resolution performance near 2039keV.
The paper examines the cross-talk, electronic noise, and resolution of the detector.
It concludes that SEGA is a viable detector for the search for neutrinoless double beta-decay.
\\

Neutrinoless double beta-decay has yet to be observed, but would show that neutrinos are majorana particles, meaning they are their own antiparticle.
These interactions have incredibly long halflives and are thus very rare.


{\large\textbf{\cite{Sangsingkeow2003183}}}
An older paper that summarizes the then recent advancements in high purity germanium detectors.
Manufacturing and fabrication methods are the focus of the improvements.
Many advancements had been made in growing large Germanium crystals that allowed for many segment detectors.
Large efficiency HPGe detectors are presented with FWHM resolutions of 2.4keV at 1.33MeV.
Multi-element arrays of Germanium detectors are introduced including the now prevalent clover array.
\\

Monolithic segmentation is an improvement covered in detail.
This technique also for further information gathering from the same germanium crystal.
Photolithography enables the use of very closely packed segmented electrical contacts.
Thus instead on one electrical output signal from a Germanium crystal, there can be many outputs.
Along with reduction in Doppler broadening, the segmentation means the addition of position information useful in tracking the path of a radiation event in the crystal.


{\large\textbf{\cite{Eberth2008283}}}
A lengthy and detailed topical review of gamma-ray tracking with Germanium detectors over the last 50 years.
Gamma-ray spectroscopy is the focus, as it has been a primarily method for probing the structure of the atomic nucleus over the decades.
The first improvements where in Germanium purity and volume.
Segmentation by separating electrical contacts to the crystals came as a later improvement.
Advances in signal processing allowed for more complex pulse shape analysis to be done on the increase in information gathered from segmentation.
The European AGATA gamma-ray tracking system is the highlighted equipment of the review.
\\

Various components and their history are discussed and compared.
Bismuth germanate and NaI(Ti) suppression shields are discussed.
A strong theoretical knowledge base is constructed for many properties of germanium semiconductor detectors.
This includes discussions from dead layers to the different ways of tiling a sphere for a detector setup.


{\large\textbf{\cite{Descovich2005535}}}
In-beam position measurements are checked against Monte Carlo simulations for highly segmented coaxial germanium detectors.
The determined resolution is 2mm in three directions for analysis on $^{90}\mbox{Zr}$ gamma transitions.
The study examines the departure of experimental results from simulated data and break down the contributions of position resolution into the various factors of the detector.
\\

The importance of position measurements is in enabling the reconstruction of the scattering path within the detecting volume.
In order to calculate the position of the first gamma interaction a Doppler correction must be made.
The identified area to improve was in selecting a zero time for pulse shape analysis.
This work is therefore very sensitive to improvements in timing information.
Time and energy information were calculated using FPGAs on board.


{\large\textbf{\cite{Mayer}}}
Semiconductor detectors eventually suffer radiation damage from usage.
This damage can be mitigated with an annealing process.
Mayer et al investigates the disorder a heavy ion brings on to the nearby lattice by helium scattering at temperatures above room temperature.
The method used gives information down to 0.2 angstroms.
A detailed account of the crystalline structure of semiconductor detectors is given.
\\

It is found that for low dosages of dopant levels the germanium crystal reorders at 180C where silicon requires 260C.
Annealing of an amorphous layer formed by high dosages occurs at 380C for germanium and 570C in silicon.
Each of the implanted germanium atom is found to displace about 3000 lattice atoms.
Diffusion and annealing both occur at significantly lower temperatures for germanium than for silicon.


{\large\textbf{\cite{Paschalis201344}}}
The Gamma-Ray Energy Tracking In-beam Nuclear Array, GRETINA, detector modules maximize solid angle coverage and tracking performance while minimizing costs in detector manufacturing.
The energy and point position of each gamma interaction can be reconstructed with high precision.
A detailed discussion on detector geometry is given to justify the design concepts of GRETINA.
The detector modules are n-type high purity germanium crystals segmented 36 times.
The testing and acceptance process is important is achieving the energy resolution of 2.2keV at 1333keV.
The digital signal processing has 10 channel inputs, each leading to a 14-bit ADC with a 100MHz sampling rate.
This information is then fed to a Xilinix FPGA which uses a trapezoidal filter to calculate energy, and leading edge and constant fraction discrimination algorithms for timing information.


{\large\textbf{\cite{Palit201290}}}
A compton suppressed germanium clover detector array with a Pixie-16 commercial digital gamma finder digitizer has been implemented at TIFR-BARC.
Gamma-gamma coincidence measurements are made using logic operations on the four clover output.
The discussion include point of improvement from moving to the digital system from the previous analog electronics.
Notably, the previous system had limitations due to heat build up causing a gain drift.
The experiments performed required a constant gain over a period of several days.
The presented digital data acquisition system has improved data recording capabilities at INGA by a factor of five.


{\large\textbf{\cite{Alberto200999}}}
Future improvements to digital acquisition systems will create self-triggering systems.
In order to achieve this digital filtering must be improved and moved to online processing.
Peak distortion, signal-to-noise ratio, energy deposition, and detector efficiency are all covered in their regard to filtering structures.
\\

The digital filters discussed are implemented on commercial Xilinix FPGAs and are compared to simulated data via Matlab and Simulink.
Noteworthy filtering algorithms include infinite impulse response low pass Butterworth III and adaptive LMS filter.
A discussion of the direct translation from VHDL shows the need for further optimization in performance driven tasks.
The VHDL translation had higher FPGA consumption and a longer logical path length than the optimized bitstream.


\subsection{Specific Detector Setups}

{\large\textbf{\cite{Chen2010261}}}
An overview of the technologies and developments for the ATLAS Liquid Argon calorimeter is given.
The scale of this project is considerable and of the largest order.
In real time there are 182,468 signals being digitized and processed on the detectors.
The discussion is centered around the upgrade being made to LAr in preparation for the increased luminosity of the LHC around 2017. The digital readout systems use both ASIC and FPGA implementations.
\\

Interesting points about the concerns of high radiation exposure are raised.
All optical links are radiation resistant due to the luminosity of the experiment environment.
Analog elements are still considered in the planned upgrades.
The voltage supply system will be upgraded and further protected from radiation.


{\large\textbf{\cite{Haefeli2006119}}}
The architecture and design of the data acquisition system for the LHCb is overviewed.
The design uses FPGAs to communicate between either digital optical or analog copper links to the Gigabit Ethernet connections.
As with all LHC related equipment the scale of the problem is rather unique.
However the logic and architecture diagrams are educational and of interest in smaller experimental setups.


{\large\textbf{\cite{Schiffer2011491}}}
A portable detector setup has been designed for the identification of radioactive materials.
The platform is scalable and uses FPGAs for the realtime integral pulse shape discrimination.
Integral calculations, pulse shape discrimination, baseline offset, time stamping, and checking for clipped or double pulses have all been implemented.
After the onboard digital data processing the results can be transmitted to a laptop for storage or further processing.
The usage of such online processing is able to reduce the necessary storage capacity by a factor of 100. The FPGA module is a Virtex 5 and is programmed using Verilog and VHDL.
The authors intended application of the platform is nuclear non-proliferation measurements.
However other industry applications such as national and airport security are mentioned.
\\[20pt]

\section{Knowledge Discovery}
Knowledge Discovery and Data Mining refers to the field of techniques and tools for finding useful trends and patterns in large volumes of data.


\pagebreak


\end{doublespacing}

%%%

\bibliography{../citationsAPSC.bib}

\end{document}
