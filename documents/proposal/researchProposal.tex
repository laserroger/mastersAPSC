\documentclass[12pt]{article}

\usepackage{hyperref}
\usepackage{setspace}
\usepackage[round]{natbib}
\bibliographystyle{plainnat}
\newcommand{\superscript}[1]{\ensuremath{^{\textrm{#1}}}}
\newcommand{\subscript}[1]{\ensuremath{_{\textrm{#1}}}}

\begin{document}
\title{Research Proposal}
\author{Andrew Valencik}
\date{\today}
\maketitle
\tableofcontents
\begin{doublespacing}

%What is the problem? 
%Why should it be done? 
%Where will it be done? 
%How will you be doing it?
%How much will it cost? 
%How long will it take? 

\pagebreak
\section{Abstract}
%The problem and its setting
%Proposed methodology to solve problem

\pagebreak
\section{Introduction}
%What is the problem? What will I study?
We want to reveal patterns and understand connections in the body of nuclear science literature to predict worthwhile areas of research.

\subsection{Motivation}
%Why is it an important question?
To find unexplored areas.
To quantify the growth in areas of research over time.
To search for interesting structures in the body of literature. The structure of citations and contributions may provide interesting insight.

To apply social mining tools to an academic dataset.
To discover.


\subsection{Review}
%What do we know already?
The body of nuclear science literature is extensive, mature, and well categorized by the National Nuclear Data Center \citep{Kurgan200603} which provides the opportunity for knowledge discovery on the literature's meta data.
This process should reveal trends in the collective scientific study of nuclear structure, processes, and detection.
Additionally, categorizing trends may provide predictive power in determining a worthwhile area of study or application of a technique.

%How will this advance our knowledge?
%Explain your choice of model.
%Significance, timeliness, and importance of project.

\pagebreak
\section{Objectives}
%Define specific goals of research (which part of general problem am I tackling?)
%Expected outcome
Determine the structure of the nuclear science literature.
Quantify the areas of growth and contribution over time.

\pagebreak
\section{Methodology}
%Critical. Define plan of action.
%How to tackle problem? Why this way?

%Visualization
%Representation of "paper"
%Form data mining problems, questions,
%Queries
%Interpretation

Classification and clustering are related approaches to organizing data elements into groups for further analysis.
Classification is the process of deciding what group a particular datum should most optimally belong to.
Clustering is the grouping of multiple data points such that those belonging to a group are more similar in some manner than those outside of that group.

Clustering can be broken into two main groups, hierarchical and partitional.
Both groups have applications in this study.
The citation structure of the literature is likely to be hierarchical in nature.
Partitional clustering should prove useful in determining the sub genres and fields of study within the body of work.


K-means clustering is a cluster analysis technique that can group data objects in $K$ clusters based on minimizes their distances with respect to cluster centroids.
K-means is a partitional clustering algorithm.
In text mining or document comparing we often use cosine similarity between term frequency vectors

\subsection{Text Mining}
Text mining is an area of analysis that focuses on extracting useful information from unstructured plain text data.
The data to analyze is often natural language text written by humans.
Examples of such data include user reviews of a product or service, customer feedback comments, emails, forum posts, or even academic journal articles.

A simple goal of text mining could be to summarize the input text.
However, this can be abstracted further to 'numericizing' text data for use in an analysis model.

%https://stackoverflow.com/questions/15173225/how-to-calculate-cosine-similarity-given-2-sentence-strings-python

\subsubsection{Sentiment Analysis}
Sentiment analysis is the extraction of subjective information, like opinions, from text data.
This can be accomplished by parsing natural language and quantifying the affectivity of keywords.


\pagebreak
\section{Equipment and Research Materials}
%Demonstrate feasibility

Computing resources, hosting, database.

\pagebreak
\section{Budget}
%Breakdown of expected expenses.

\pagebreak
\section{Timetable}
%Expected date of completion for each major step in research project
%Realistic for 2 year project

%%% That's all folks
\pagebreak
\end{doublespacing}

\bibliography{../citationsAPSC.bib}

\end{document}
