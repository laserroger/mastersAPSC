\documentclass[12pt]{article}

\usepackage{hyperref}
\usepackage[round]{natbib}
\bibliographystyle{plainnat}

\newcommand{\superscript}[1]{\ensuremath{^{\textrm{#1}}}}
\newcommand{\subscript}[1]{\ensuremath{_{\textrm{#1}}}}
%\newcommand{\th}[0]{\superscript{th}}
%\newcommand{\st}[0]{\superscript{st}}
%\newcommand{\nd}[0]{\superscript{nd}}
%\newcommand{\rd}[0]{\superscript{rd}}

\begin{document}

\title{Digital Signal Processing in Nuclear Physics with Field Programmable Gate Arrays}

\author{Andrew Valencik}

\date{\today}

\maketitle

%%%%

\section{Field Programmable Gate Arrays}
This section is about FPGAs
\\[20pt]

{\large\textbf{\cite{currentState}}}

The article is an industry produced review of FPGAs in the nuclear power sector. It has a good and thorough introduction to FPGA technology, including a discussion of hardware aspects such as CMOS, chip archetecture, different implementations (fuse, SRAM, antifuse), electrical and mechanical propertires, and the reliability of FPGAs.
\\
A good deal of the discussion focuses on the design and development of applications for FPGAs. The development process includes aspects from software design along with concerns and practices in hardware design. Finally, due to the nature of a nuclear power plant, an indepth analysis is given on the concerns of safety and risks associated with a FPGA system. This includes suggestions such as triple modular redudancy (TMR), which is covered in detail, using error detection and correction codes, and diverse methods of implementation.
\\[20pt]

{\large\textbf{\cite{Wirthlin}}}

The embedded applications, field reprogrammability, low non-reocurring engineering cost, and the short design cycle of FPGAs means they end up in a large range of environments. This study focuses on FPGAs in high radiation environments and draws largely off the knowledge base gained from the use of FPGA in space. 
\\
The defined bitstrem, flip-flops, registers of a FPGS are sensitive to heavy ion and proton single event upsets that might occur near a radiation detector. An expensive option to protect against single evenet upsets is to use triple modular redundancy. This involves duplicating logic components and comparing the results after various operations to protecting againt the failure of a logic unit. Another method is configuration scrubbing by reflashing the bitstream on the fly, to avoid the build up of failures. It is likely that the efforts to migate radiation effects in space will be useful in particle physics experiments as well. There is an ungoing effort to used FPGAs in the liquird argon calorimeter at ATLAS. 
\\[20pt]

{\large\textbf{\cite{Cromaz}}}

For the GRETINA spectrometer a multichannel digital signal processing board has been custom built using FPGAs. The board has 12-bit resolution and is capable of digitizing continuously at 100MHz. The FPGA on the board derives an energy, leading-edge time, and constant-fraction time. The board discussed was a prototype to the actual boards which use 10 channels and 14-bit ADCs which should improve issues with differential non-linearity. The FPGA used in the production board is an commercially available one, the Xilinx Spartan XC3S5000.
\\[20pt]


{\large\textbf{\cite{Ugur}}}

One of the most important aspects of particle identification experiments is the digitisation of time, amplitude and charge data from detectors. These conversions are mostly undertaken with Application Specific Integrated Circuits (ASICs). However, recent developments in Field Pro- grammable Gate Array (FPGA) technology allow us to use commercial electronic components for the required Front-End Electronics (FEE) and to do the digitisation in the FPGA. It is possible to do Time-of-Flight (ToF), Time-over-Threshold (ToT), amplitude and charge measurements with converters implemented in FPGA. 
\\[20pt]


%%%%%%%%%%

\section{Nuclear Physics}


{\large\textbf{\cite{Mayer}}}

The orientation dependence of the backscattered yield of 1.0-MeV helium ions has been used to investigate the lattice characteristics of silicon and germanium implanted at room temperature with 40-keV heavy ions (Ga, As, Sb, In, P). 
\\[20pt]


\bibliography{citationsAPSC.bib}

\end{document}
